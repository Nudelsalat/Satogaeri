This bachelor thesis is about a Japanese logic puzzle called Satogaeri. First the thesis will give some general information about Satisfiability Modulo Theories which will be the main focus throughout the thesis. It covers the development of a solver, a generator and a user interface which allows to play and create Satogaeri puzzles. In the later chapters the thesis will give some statistics about different logics that were used and how the solver and generator hold up time-wise. It concludes with the goals that were achieved and how different parts of the project may be improved in the future.